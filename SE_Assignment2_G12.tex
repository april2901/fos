FoS(Focus on Speaking)

\textbf{Group 12}

https://github.com/april2901/ai-assistant-for-presentation

Sangyoon Kwon\\
\emph{Department of Computer Science\\
Backend Development\\
}Seoul, Republic of Korea\\
is0110@hanyang.ac.kr

Hyeyun Kwon\\
\emph{Department of Information Systems\\
Frontend Development\\
}Seoul, Republic of Korea\\
herakwon1124@hanyang.ac.kr\\
Dohoon Kim\\
\emph{Department of Computer Science\\
Backend Development\\
}Seoul, Republic of Korea\\
april2901@hanyang.ac.kr

Seohyun Kim\\
\emph{Department of Information Systems\\
Frontend Development\\
}Seoul, Republic of Korea\\
dianwls0326@hanyang.ac.kr\\
Daeun Lee\\
\emph{Division of Business Administratio}n\\
\emph{UI Design, PM\\
}Seoul, Republic of Korea\\
shinran2929@hanyang.ac.kr

Minhyuk Jang\\
\emph{Division of Business Administration\\
UI Design, PM\\
}Seoul, Republic of Korea\\
jmh12230@hanyang.ac.kr

\emph{Abstract}--- In modern business and educational environments,
meetings and presentations are key means of communication, and their
importance continues to grow. However, presenters and participants often
experience cognitive overload as they manage speech delivery, script
reference, slide transitions, and time management while also trying to
follow complex discussion flows and decisions. This leads to topic
drift, loss of focus, and unclear outcomes. To address these issues,
this project proposes an \textbf{LG display--linked real-time meeting AI
prompter} that extends a traditional teleprompter into an active,
context-aware assistant. The system listens to participants' speech,
interprets the meeting context in real time, and presents the next
required information---such as agenda structure, current topic,
decisions, and action items---on LG displays, while providing a private
coaching dashboard for the presenter or host. Core features include
real-time STT, flexible speech-to-script matching, keyword omission
detection, a real-time feedback dashboard, agenda visualization,
decision and action-item extraction, and fact-check widgets. A Meeting
Summary Report summarizes key topics, ideas, decisions, and action items
after the session. Through these functions, the project aims to improve
both individual presentation quality and overall meeting efficiency, and
to explore integration within LG's smart office ecosystem.\\
\strut \\
\emph{\textbf{Keywords---Speech Recognition, Real-Time STT, Script
Synchronization, Real-Time Teleprompter, Slide Automation, Agenda
Tracking, Presentation Feedback, Human-Computer Interaction, Meeting
Intelligence}}

Role Assignment -

{\def\LTcaptype{none} % do not increment counter
\begin{longtable}[]{@{}
  >{\raggedright\arraybackslash}p{(\linewidth - 4\tabcolsep) * \real{0.1188}}
  >{\raggedright\arraybackslash}p{(\linewidth - 4\tabcolsep) * \real{0.0953}}
  >{\raggedright\arraybackslash}p{(\linewidth - 4\tabcolsep) * \real{0.2644}}@{}}
\toprule\noalign{}
\begin{minipage}[b]{\linewidth}\raggedright
\textbf{Roles}
\end{minipage} & \begin{minipage}[b]{\linewidth}\raggedright
\textbf{Name}
\end{minipage} & \begin{minipage}[b]{\linewidth}\raggedright
\textbf{Task description and etc.}
\end{minipage} \\
\midrule\noalign{}
\endhead
\bottomrule\noalign{}
\endlastfoot
\textbf{User} & Daeun Lee, Minhyuk Jang & Tests the prototype from the
user's perspective, focusing on interface usability, speech
synchronization accuracy, and overall user experience. Provides
qualitative feedback for refinement. \\
\textbf{Customer} & LG Electronics (Assumed Client) & Defines
requirements for smart office presentation support software and
evaluates its feasibility for integration with LG's webOS-based business
ecosystem. \\
\textbf{Software Developer} & Sangyoon Kwon, Dohoon Kim (Backend),
Hyeyun Kwon, Seohyun Kim (Frontend) & Responsible for system
implementation including backend server logic, database management, API
communication, and frontend interface development. Ensures real-time
synchronization and stable slide automation. \\
\textbf{Development Manager\& UI Designer} & Daeun Lee, Minhyuk Jang &
Oversees project planning, documentation, and communication between
development teams. Manages task allocation, schedule tracking, design of
interface and quality assurance. \\
\end{longtable}
}

\section{Introduction}\label{introduction}

\textbf{A. Challenges in Modern Presentations and Meetings}

In professional and academic settings, presentations and meetings have
become essential tools for sharing ideas and making decisions. Yet
presenters and facilitators must simultaneously handle speech delivery,
script reference, slide transitions, and time tracking, while
participants struggle to follow complex discussion flows and remember
key points. This often causes interruptions in the presentation flow,
omission of important content, topic drift during discussions, and
unclear conclusions, ultimately reducing communication efficiency.

B. \textbf{Limitations of Existing Solutions}

Existing tools such as teleprompters, timers, and subtitle features
mainly provide static information or simple transcription. They are
useful for displaying text but do not actively intervene in real time to
prevent topic drift or support decision alignment. Many AI-based meeting
services focus on post-meeting summaries or minutes, which help review
what happened but do not improve the efficiency of the meeting while it
is in progress. As a result, core issues such as cognitive overload and
live meeting inefficiency remain unresolved.

\textbf{C. Project Goals and Proposed Solution}

This project proposes an integrated support system that covers
preparation, live delivery, and post-meeting feedback. The core concept
is an \textbf{``LG Display--Linked Real-Time Meeting AI Prompter.''} The
system continuously listens to meeting audio, analyzes the semantic
context of each utterance, and surfaces what the meeting needs next:
agenda structure, current topic, decisions, action items, and fact-check
results. At the same time, it provides a private teleprompter and
coaching dashboard for the presenter or host, helping with script
tracking, omission alerts, pacing, and gesture suggestions. The ultimate
goal is to reduce cognitive load and improve meeting focus and
decision-making speed.

\textbf{D. Dual-Screen Architecture for LG Displays}\\
To support both individual coaching and shared awareness, the system
adopts a dual-screen architecture. \textbf{Screen 1 (Presenter
Dashboard)} is shown on the presenter's personal device (e.g., LG Gram)
and provides a private teleprompter, omission alerts, pace metrics, and
AI suggestions. \textbf{Screen 2 (Shared Meeting Board)} is shown on LG
signage, LG One:Quick, or conference room TVs and visualizes slides, the
real-time agenda map, decisions and action items, and fact-check widgets
for all participants. This separation allows the presenter to receive
rich guidance without overwhelming the audience, while participants
share a clear view of where the meeting is and what has been decided.

\section{Requirements}\label{requirements}

\textbf{A. Before Presentation}

This phase focuses on the preparation process before a presenter begins
their presentation. Users interact with the system to upload materials,
create a script, and adjust content to fit the presentation environment.

\subsubsection{\texorpdfstring{\textbf{Slide-Script Alignment
Recognition and
Consulting}}{Slide-Script Alignment Recognition and Consulting}}\label{slide-script-alignment-recognition-and-consulting}

When a user uploads a PPTX or PDF file, the system extracts textual and
visual elements using python-pptx and the Google Vision API (OCR). A
multimodal LLM processes these elements to interpret textual and
graphical contexts, generating a coherent draft script for each slide.

\begin{itemize}
\item
  Acceptance Criteria
\end{itemize}

• OCR text recognition accuracy ≥ 95\%\\
• Draft script grammatical accuracy ≥ 95\%\\
• Slide--text coherence ≥ 95\%

\begin{itemize}
\item
  Input \& Output
\end{itemize}

• Input\textbf{:} Presentation file (.pptx, .pdf)\\
• Output\textbf{:} Structured text/image metadata, draft script (3--6
sentences per slide)

\begin{itemize}
\item
  Constraints
\end{itemize}

• Max 100 slides\\
• Max file size 200 MB\\
• Max 10 images per slide\\
• Supported formats: PPTX, PDF

\subsubsection{\texorpdfstring{\textbf{Presentation Environment-Specific
Script
Adjustment}}{Presentation Environment-Specific Script Adjustment}}\label{presentation-environment-specific-script-adjustment}

The system adjusts the script's vocabulary level, tone, and length based
on the audience type (non-expert / practitioner / expert), target
presentation time, and speaker's pace. Using TensorFlow.js-based vision
models, audience facial expressions are analyzed every 3 seconds to
compute a ``focus score'' (0--100). When time is running short or
audience engagement decreases, an LLM provides real-time summaries or
interactive remarks.

\begin{itemize}
\item
  Acceptance Criteria
\end{itemize}

• Script adjustment time ≤ 5s\\
• Focus detection accuracy ≥ 85\%\\
• Timing deviation ≤ ±5\%

• Script fluency ≥ 90\%

\begin{itemize}
\item
  Input \& Output
\end{itemize}

• Input\textbf{:} Audience type, target duration, speech rate(WPM),
tone, audience video data\\
• Output\textbf{:} Adjusted script (.txt), recommended timing table,
real-time teleprompter feedback

\begin{itemize}
\item
  Constraints
\end{itemize}

• Camera ≥ 720p\\
• Max 10 audience members detectable\\
• LLM request frequency ≤ 1 per 10 s\\
• Presentation ≤ 60 min

\textbf{B. Live Delivery Phase}

This phase involves real-time interaction between the user and the
system during the actual presentation. The system detects the
presenter\textquotesingle s speech and performs instant support tasks
like synchronization, feedback, and suggestions.

\subsubsection{\texorpdfstring{\textbf{Real-time Teleprompter(Meeeting
Mode
Support)}}{Real-time Teleprompter(Meeeting Mode Support)}}\label{real-time-telepromptermeeeting-mode-support}

The system transcribes the presenter's and participants' speech in real
time using Google Cloud or Naver Clova STT and aligns it with the
prepared script via KoSentence-BERT semantic similarity. The current
sentence is visually highlighted on the teleprompter. In meeting mode,
the STT pipeline can capture speech from multiple people in the room as
a single mixed audio stream; downstream modules operate on this combined
transcript.

\begin{itemize}
\item
  Acceptance Criteria
\end{itemize}

• Speech--script synchronization delay ≤ 1 s\\
• Highlight accuracy ≥ 95\%\\
• Alignment deviation ≤ 1 sentence

\begin{itemize}
\item
  Input \& Output
\end{itemize}

• Input\textbf{:} Microphone audio (.wav, .mp3, ≥16 kHz), script file
(.txt)

• Output\textbf{:} Real-time highlighted script text and STT logs

\begin{itemize}
\item
  Constraints
\end{itemize}

• Session length ≤ 60 min\\
• STT throughput ≥ 50 words/s\\
• API cost ≈ \$0.006/min

\subsubsection{\texorpdfstring{\textbf{Automatic Slide
Transition}}{Automatic Slide Transition}}\label{automatic-slide-transition}

Through the Microsoft PowerPoint COM API, slides are automatically
advanced when the script reaches predefined transition points.

\begin{itemize}
\item
  Acceptance Criteria
\end{itemize}

• Transition latency ≤ 0.5 s\\
• Transition accuracy ≥ 95\%\\
• Failure rate ≤ 5\%

\begin{itemize}
\item
  Input \& Output
\end{itemize}

• Input\textbf{:} Slide file (.pptx), predefined transition IDs

• Output\textbf{:} Automatically advanced slide display

\begin{itemize}
\item
  Constraints
\end{itemize}

• Max 100 slides\\
• Requires PowerPoint 2016 or later

\subsubsection{\texorpdfstring{\textbf{Flexible Speech-to-Script
Matching}}{Flexible Speech-to-Script Matching}}\label{flexible-speech-to-script-matching}

The system maintains synchronization even when speech deviates lexically
from the script. Primary matching uses KoSentence-BERT vector similarity
(threshold ≥ 0.8), followed by secondary LLM-based contextual
verification if necessary.

\begin{itemize}
\item
  Acceptance Criteria
\end{itemize}

• Matching success rate ≥ 90\%\\
• False match rate ≤ 5\%\\
• Matching latency ≤ 0.3 s per sentence

\begin{itemize}
\item
  Input \& Output
\end{itemize}

• Input\textbf{:} STT transcript, script text

• Output\textbf{:} Matched sentence ID and highlight position

\begin{itemize}
\item
  Constraints
\end{itemize}

• LLM call limit ≤ 1 per second\\
• STT buffering ≤ 5 s

\subsubsection{\texorpdfstring{\textbf{Key Content Omission
Detection}}{Key Content Omission Detection}}\label{key-content-omission-detection}

\subsubsection{Detects missing predefined key phrases using cosine
similarity (threshold ≥ 0.75) and alerts the presenter within 3
seconds.}\label{detects-missing-predefined-key-phrases-using-cosine-similarity-threshold-0.75-and-alerts-the-presenter-within-3-seconds.}

\begin{itemize}
\item
  Acceptance Criteria
\end{itemize}

• Detection precision ≥ 95\%\\
• False alarm ≤ 5\%\\
• Alert delay ≤ 2 s

\begin{itemize}
\item
  Input \& Output
\end{itemize}

• Input\textbf{:} STT transcript, key phrase list (≤50)

• Output\textbf{:} Omission alert and log file

\begin{itemize}
\item
  Constraints
\end{itemize}

• Max 500 sentences compared\\
• STT buffering interval: 5 s

\subsubsection{\texorpdfstring{\textbf{Real-time Script
Reconstruction}}{Real-time Script Reconstruction}}\label{real-time-script-reconstruction}

When omissions are detected, missing segments are asynchronously sent to
an LLM that generates supplementary sentences within 5 seconds. Approved
sentences are integrated into the script in real time.

\begin{itemize}
\item
  Acceptance Criteria
\end{itemize}

• Supplement generation ≤ 5 s\\
• Contextual coherence ≥ 90\%\\
• Integration success ≥ 90\%

\begin{itemize}
\item
  Input \& Output
\end{itemize}

• Input\textbf{:} Missing sentence ID, context text, LLM API key

• Output\textbf{:} Supplementary sentence, updated script

\begin{itemize}
\item
  Constraints
\end{itemize}

• Max 10 API calls per minute\\
• Sentence length ≤ 100 characters

\subsubsection{\texorpdfstring{\textbf{Real-time Presenter
Dashboard}}{Real-time Presenter Dashboard}}\label{real-time-presenter-dashboard}

The system visualizes metrics such as words per minute (WPM), voice
volume, and progress rate using the Web Audio API, and applies
TensorFlow Lite models for basic emotion recognition (e.g.,
tension/calmness).

\begin{itemize}
\item
  Acceptance Criteria
\end{itemize}

• Data refresh ≤ 2 s\\
• Emotion inference error ≤ ±5\%\\
• Visualization accuracy ≥ 95\%

\begin{itemize}
\item
  Input \& Output
\end{itemize}

• Input\textbf{:} Audio stream, STT logs

• Output\textbf{:} Live dashboard showing progress, WPM, emotional state

\begin{itemize}
\item
  Constraints
\end{itemize}

• Sampling rate ≥ 16 kHz\\
• Dashboard latency ≤ 1 s

\subsubsection{\texorpdfstring{\textbf{Speech Gesture
Suggestions}}{Speech Gesture Suggestions}}\label{speech-gesture-suggestions}

Before the presentation, the system analyzes slide images with a
multimodal LLM to detect key visual elements (graphs, photos, diagrams)
and map them to related keywords. During speech, when such keywords
appear in STT, gesture icons (e.g., pointing, emphasis) are displayed on
the teleprompter.

\begin{itemize}
\item
  Acceptance Criteria
\end{itemize}

• Suggestion latency ≤ 2 s\\
• Gesture relevance ≥ 85\%\\
• Recognition accuracy ≥ 90\%

\begin{itemize}
\item
  Input \& Output
\end{itemize}

• Input\textbf{:} Slide images, script keywords

• Output\textbf{:} Gesture icons displayed on teleprompter

\begin{itemize}
\item
  Constraints
\end{itemize}

• Max 3 visual mappings per keyword\\
• Display duration: 2--3 s

\subsubsection{Real-Time Context and Intent
Tagging}\label{real-time-context-and-intent-tagging}

The system analyzes STT text in real time and classifies each utterance
according to its intent and type. This information is used as the
foundation for the agenda map, decision board, and fact-check triggers.

\begin{itemize}
\item
  Supported Tags
\end{itemize}

• General comment

• Idea proposal

• Negative/Positive feedback

• Decision

• Request / Action Item

• Question

• Fact-check request

\begin{itemize}
\item
  Acceptance Criteria
\end{itemize}

• all utterances receive at least one tag from the predefined set

• Intent classification accuracy ≥ 85\% on test samples

\begin{itemize}
\item
  Input \& Output
\end{itemize}

• Input: STT transcript segmented into utterances

• output: Tagged utterance stream (text + tag(s) + timestamp)

\begin{itemize}
\item
  Constraints
\end{itemize}

• Classification latency ≤ 0.5s per utterance

• the tagging model must operate within the WebSocket round-trip time
budget

\subsubsection{Real-Time agenda Map}\label{real-time-agenda-map}

Using the tagged utterance stream, the system builds a real-time agenda
map to prevent topic deviation and improve shared awareness. Each
emerging topic is registered as a node in a network graph displayed on
\textbf{Screen 2}

\begin{itemize}
\item
  Acceptance Criteria
\end{itemize}

• Screen 2 displays a live STT log and its mapping to agenda nodes\\
• Utterances tagged as ``Idea,'' ``Decision,'' or ``Action Item'' are
grouped under appropriate agenda nodes\\
• Nodes are color-coded by agenda type

• The active topic is clearly highlighted

• Node detail view shows STT snippet and timestamp

\begin{itemize}
\item
  Input \& Output
\end{itemize}

• Input\textbf{:} Tagged utterance stream (from Requirement 8), semantic
embeddings

• Output\textbf{:} Real-time agenda network graph on Screen2

\begin{itemize}
\item
  Constraints
\end{itemize}

• Graph update interval ≤ 2s\\
• Max 30 agenda nodes per session

\subsubsection{Dual-Screen synchronizatino between Presenter Shared
Display}\label{dual-screen-synchronizatino-between-presenter-shared-display}

The system keeps Screen 1(presenter Dashboard) and Screen 2 synchronized
while respecting privacy boundaries

\begin{itemize}
\item
  Acceptance Criteria
\end{itemize}

• Slide transition latency between Screen 1 and Screen 2 ≤ 0.5 s\\
• No private elements (teleprompter text, omission alerts, AI
suggestions) appear on Screen 2

\begin{itemize}
\item
  Input \& Output
\end{itemize}

• Input\textbf{:} Slide control events, layout state, synchronization
messages

• Output\textbf{:} Consistent view of the current slide and agenda state
on Screen 2

\begin{itemize}
\item
  Constraints
\end{itemize}

• WebSocket synchronization interval ≤ 1s

\subsubsection{Real-time Decisions and Action Item
Widget}\label{real-time-decisions-and-action-item-widget}

The system captures utterances tagged as decision or action item and
surfaces them in a dedicated widget on screen 2, often placed below or
beside the agenda map.

\begin{itemize}
\item
  Acceptance Criteria
\end{itemize}

• Detection coverage for decision-like and action-like utterances ≥ 90\%
on test scenarios\\
• New items appear in the list within 2 seconds of the utterance

\begin{itemize}
\item
  Input \& Output
\end{itemize}

• Input\textbf{:} Tagged utterance stream (Decision / Action Item tags)

• Output\textbf{:} Real-time decision \& action-item list on Screen 2

\begin{itemize}
\item
  Constraints
\end{itemize}

• Max 100 items per session

• Each item stored with its text content and timestamp, with a link to
the original transcript segment

\subsubsection{Real-Time Fact-Check and Research
Widget}\label{real-time-fact-check-and-research-widget}

When the system detects a \textbf{Fact-check request} tag, it triggers a
lightweight research pipeline (RAG or web search) and surfaces the
result on Screen 2

\begin{itemize}
\item
  Acceptance Criteria
\end{itemize}

• Successful keyword extraction for at least 90\% of fact-check
requests\\
• Research results displayed within 5 seconds

\begin{itemize}
\item
  Input \& Output
\end{itemize}

• Input\textbf{:} Tagged utterance stream (Fact-check request tag),
knowledge base or web search API

• Output\textbf{:} Fact-check result widget with short answer and source
link(s)

\begin{itemize}
\item
  Constraints
\end{itemize}

• Max 30 fact-check queries per session

• Each query result limited to brief, conference-friendly summaries

\subsubsection{\texorpdfstring{\textbf{C. Post-Presentation
Phase}}{C. Post-Presentation Phase}}\label{c.-post-presentation-phase}

\subsubsection{This phase involves the user receiving feedback on their
presentation and conducting a Q\&A session after the presentation has
concluded.}\label{this-phase-involves-the-user-receiving-feedback-on-their-presentation-and-conducting-a-qa-session-after-the-presentation-has-concluded.}

\subsubsection{Q\&A Auto-Response}\label{qa-auto-response}

In Q\&A mode, the system uses Retrieval-Augmented Generation (RAG) to
search a pre-built database and generate 2--3 candidate answers, each
referencing supporting slides or pages.

\begin{itemize}
\item
  Acceptance Criteria
\end{itemize}

• Answer generation ≤ 5 s\\
• Relevance score ≥ 0.85\\
• Slide reference accuracy ≥ 98\%

\begin{itemize}
\item
  Input \& Output
\end{itemize}

• Input\textbf{:} Question (speech/text), presentation DB (JSON)

• Output\textbf{:} 2--3 candidate answers with referenced slides

\begin{itemize}
\item
  Constraints
\end{itemize}

• Max 20 questions per session\\
• Max 300 tokens per answer\\
• RAG cosine similarity ≥ 0.8

\subsubsection{Presentation and Meeting Analysis
Report}\label{presentation-and-meeting-analysis-report}

\subsubsection{After the meeting or presentation, the system
automatically analyzes collected data (speech logs, agenda map,
decisions, action items, and referenced research results) and generates
a Meeting Summary Report. This report covers time management, speech
habits, content delivery, and meeting outcomes such as major topics,
ideas, decisions, action items, and external facts referenced during the
discussion.}\label{after-the-meeting-or-presentation-the-system-automatically-analyzes-collected-data-speech-logs-agenda-map-decisions-action-items-and-referenced-research-results-and-generates-a-meeting-summary-report.-this-report-covers-time-management-speech-habits-content-delivery-and-meeting-outcomes-such-as-major-topics-ideas-decisions-action-items-and-external-facts-referenced-during-the-discussion.}

\begin{itemize}
\item
  Acceptance Criteria
\end{itemize}

• Report generation ≤ 10 s\\
• Analysis accuracy ≥ 95\%\\
• User satisfaction ≥ 4.2/5.0

\begin{itemize}
\item
  Input \& Output
\end{itemize}

• Input\textbf{:} Speech logs, slide transitions, emotion data, agenda
map, decision/action-item list, fact-chcek logs

• Output\textbf{:} Analysis report (.html,.pdf)

\begin{itemize}
\item
  Constraints
\end{itemize}

• Max presentation time: 60 min\\
• Max 100,000 words processed

\section{\texorpdfstring{\textbf{Version control system}
}{Version control system }}\label{version-control-system}

To manage source code and documentation, a version control system based
on \textbf{Git} was established.\\
A public repository named \textbf{``ai-assistant-for-presentation''} was
created on \textbf{GitHub}.

The following source code and documents have been uploaded to the
repository:

\begin{itemize}
\item
  All backend and frontend source code
\item
  Shared documents including this file
  (\emph{project\_documentation.md}) and other design files
\item
  Configuration files and project-related assets
\end{itemize}

All team members (development, project management, and UI/UX design)
have been granted access to the repository.\\
For efficient GitHub management, the team will adopt a \textbf{branch
management strategy} as illustrated in the diagram below.

\includegraphics[width=2.86585in,height=2.05427in,alt={스크린샷, 도표, 라인, 그래프이(가) 표시된 사진 AI 생성 콘텐츠는 정확하지 않을 수 있습니다.}]{media/image1.png}

\textbf{IV. Development Environment}

\emph{\textbf{A. Choice of Software Development Platform}}

\begin{enumerate}
\def\labelenumi{\arabic{enumi})}
\item
  \emph{Platform Selection and Rationale}
\end{enumerate}

\begin{quote}
The project adopts a web-based client--server architecture as its
primary development and deployment platform. The web environment ensures
platform-independent accessibility without requiring users to install
additional software, while providing seamless integration with
cloud-based APIs such as Google Cloud Speech-to-Text and
large-language-model (LLM) services. Modern web technologies, including
WebSockets, Web Audio API, and TensorFlow.js, enable the implementation
of essential real-time features such as live teleprompting and
synchronized feedback dashboards.
\end{quote}

\begin{enumerate}
\def\labelenumi{\arabic{enumi})}
\setcounter{enumi}{1}
\item
  \emph{Programming Languages and Rationale}
\end{enumerate}

\begin{itemize}
\item
  Backend: Python (version 3.11 or higher) using FastAPI for
  asynchronous API and WebSocket communication. Python is the de facto
  standard for AI and machine learning workflows, offering a mature
  ecosystem that includes python-pptx, sentence-transformers, and
  google-cloud-speech libraries.
\item
  Frontend: TypeScript/JavaScript (Node.js 20 or higher) with React
  18.2. JavaScript is the only natively supported browser language and
  is indispensable for client-side interaction. React combined with
  TypeScript supports modular, maintainable UI components and ensures
  type safety.
\end{itemize}

\begin{enumerate}
\def\labelenumi{\arabic{enumi}.}
\setcounter{enumi}{2}
\item
  \emph{Cost Estimation}\\
  The estimated total development cost is approximately USD 30.00, as
  summarized below:
\end{enumerate}

\begin{itemize}
\item
  Hardware: Personal laptops (MacBook Air/Pro) -- USD 0.00
\item
  Software and IDE: Visual Studio Code (free), Cursor Pro (USD 20 per
  month)
\item
  Cloud Services and APIs: AWS Free Tier, GCP STT and Vision API, OpenAI
  GPT Realtime Mini (approx. USD 10)
\end{itemize}

\begin{enumerate}
\def\labelenumi{\arabic{enumi}.}
\setcounter{enumi}{3}
\item
  \emph{Development Environment Details}
\end{enumerate}

\begin{itemize}
\item
  Operating Systems: Windows 11, macOS 14 (Sonoma)
\item
  IDEs: Visual Studio Code (v1.90 or higher), Cursor (v1.7 or higher)
\item
  Version Control: Git (v2.39 or higher) and GitHub public repository
  (``ai-assistant-for-presentation'')
\item
  Backend Stack: Python 3.11+, FastAPI 0.110+, PostgreSQL 16 (Render
  hosted)
\item
  Frontend Stack: Node.js 20.10+, npm 10.2+, React 18.2+, TypeScript
  5.2+
\item
  Major Libraries: python-pptx, sentence-transformers, websockets,
  TensorFlow.js, Web Audio API
\item
  Hardware Resources: Three personal laptops (two MacBook Airs, one
  MacBook Pro) used for development and testing.
\end{itemize}

\begin{enumerate}
\def\labelenumi{\arabic{enumi}.}
\setcounter{enumi}{4}
\item
  \emph{Use of Commercial Cloud Platforms}
\end{enumerate}

\begin{itemize}
\item
  Google Cloud Platform (GCP): Utilized for Speech-to-Text and Vision
  OCR services to enable high-accuracy transcription and slide text
  extraction. Services operate within free-tier quotas.
\item
  Amazon Web Services (AWS): EC2 for backend deployment, S3 for file
  storage, RDS for database hosting, and Route 53 for domain management.
  Free-tier services are used for prototype deployment and
  demonstration.
\end{itemize}

\subsubsection{\texorpdfstring{\textbf{B. Software in
Use}}{B. Software in Use}}\label{b.-software-in-use}

Several existing software solutions and research studies were referenced
in designing the system:

\begin{itemize}
\item
  PromptSmart (VoiceTrack): A commercial teleprompter offering real-time
  voice tracking. The proposed system extends its capabilities by adding
  semantic matching, omission detection, and automated slide control.
\item
  Microsoft PowerPoint (Live Subtitles): Provides speech transcription
  but lacks contextual synchronization with scripts and automated slide
  transitions.
\end{itemize}

These benchmarks highlight the project's improvements in real-time
adaptivity and AI-driven presentation assistance.

\subsubsection{\texorpdfstring{\textbf{C. Task
Distribution}}{C. Task Distribution}}\label{c.-task-distribution}

{\def\LTcaptype{none} % do not increment counter
\begin{longtable}[]{@{}
  >{\raggedright\arraybackslash}p{(\linewidth - 4\tabcolsep) * \real{0.1285}}
  >{\raggedright\arraybackslash}p{(\linewidth - 4\tabcolsep) * \real{0.1007}}
  >{\raggedright\arraybackslash}p{(\linewidth - 4\tabcolsep) * \real{0.2540}}@{}}
\toprule\noalign{}
\begin{minipage}[b]{\linewidth}\centering
\textbf{Role}
\end{minipage} & \begin{minipage}[b]{\linewidth}\centering
\textbf{Members}
\end{minipage} & \begin{minipage}[b]{\linewidth}\centering
\textbf{Responsibilities}
\end{minipage} \\
\midrule\noalign{}
\endhead
\bottomrule\noalign{}
\endlastfoot
Backend Development & Sangyoon Kwon, Dohoon Kim & System architecture
design, FastAPI server and WebSocket implementation, database schema
(PostgreSQL), AI logic integration (STT, LLM, BERT), cloud deployment
(AWS, Render) \\
Frontend Development & Hyeyun Kwon, Seohyun Kim & React-based UI
implementation, client-side state management, real-time dashboard (Web
Audio API), teleprompter interface, client-side AI (TensorFlow.js) \\
Project Management \& UI Design & Daeun Lee, Minhyuk Jang & Project
planning and scheduling, UI/UX design (Figma), documentation and VCS
management, user testing and feedback analysis \\
\end{longtable}
}

\section{\texorpdfstring{\textbf{Specification}}{Specification}}\label{specification}

\subsubsection{\texorpdfstring{\textbf{Requirement 1. Real-Time
Teleprompter}}{Requirement 1. Real-Time Teleprompter}}\label{requirement-1.-real-time-teleprompter}

This process involves a tightly coordinated real-time loop between the
client (web browser) and the server (Python backend) through WebSocket
communication.

\begin{enumerate}
\def\labelenumi{\arabic{enumi}.}
\item
  \textbf{Client Initialization}\\
  When the user presses ``Start Presentation,'' the React frontend
  requests microphone access using navigator.mediaDevices.getUserMedia()
  and establishes a secure WebSocket (wss://) connection to the backend
  API server.\\
  The Web Audio API initializes an AudioContext and ScriptProcessorNode
  to capture raw audio chunks.
\item
  \textbf{Client-Side Real-Time Audio Streaming}\\
  The ScriptProcessorNode continuously triggers onaudioprocess events
  (e.g., every 500 ms).\\
  Each raw audio buffer (16-bit PCM) is sent to the backend server
  through WebSocket.
\item
  \textbf{Server-Side STT and Synchronization}\\
  The backend receives the audio chunks and streams them to the Google
  Cloud Speech-to-Text API.\\
  The API returns \emph{interim} (fast but less accurate) and
  \emph{final} (slower but more accurate) transcripts.\\
  When a final sentence is received, it is appended to the full
  transcript of the session.\\
  The backend then calls the FlexibleSpeechMatcher service to locate the
  new currentSentenceIndex within the user's script.
\item
  \textbf{Server Broadcast}\\
  The server immediately sends a WebSocket message to the client:\\
  \{ "action": "UPDATE\_TELEPROMPTER", "index": currentSentenceIndex \}
\item
  \textbf{Client Update}\\
  The frontend WebSocket listener receives the message and updates the
  highlighted text accordingly, scrolling the teleprompter to the
  current index in real time.
\end{enumerate}

\subsubsection{\texorpdfstring{\textbf{Requirement 2. Flexible
Speech-to-Script
Matching}}{Requirement 2. Flexible Speech-to-Script Matching}}\label{requirement-2.-flexible-speech-to-script-matching}

This algorithm implements a hybrid matching mechanism combining fast
vector similarity and fallback large-language-model (LLM) validation.

\textbf{Pseudocode Overview}

{\def\LTcaptype{none} % do not increment counter
\begin{longtable}[]{@{}
  >{\raggedright\arraybackslash}p{(\linewidth - 0\tabcolsep) * \real{0.4812}}@{}}
\toprule\noalign{}
\begin{minipage}[b]{\linewidth}\raggedright
SIMILARITY\_THRESHOLD = 0.75

SEARCH\_WINDOW = 5

LLM\_VALIDATION\_THRESHOLD = 0.60

Function FindCurrentPosition(fullTranscript, scriptSentences,
lastIndex):

latestTranscript = GetLastNWords(fullTranscript, 10)

transcriptVector = KoSentenceBERT.encode(latestTranscript)

searchStart = lastIndex

searchEnd = min(lastIndex + SEARCH\_WINDOW, len(scriptSentences))

searchWindow = scriptSentences{[}searchStart : searchEnd{]}

bestMatchIndex = -1

highestSimilarity = 0

\# Step 1: Fast vector similarity

for index, sentence in enumerate(searchWindow):

similarity = CosineSimilarity(transcriptVector, sentence.vector)

if similarity \textgreater{} highestSimilarity:

highestSimilarity = similarity

bestMatchIndex = searchStart + index

\# Step 2: Omission check

if bestMatchIndex \textgreater{} lastIndex:

CheckForOmissions(lastIndex, bestMatchIndex, scriptSentences)

if highestSimilarity \textgreater= SIMILARITY\_THRESHOLD:

return bestMatchIndex

\# Step 3: LLM fallback validation

if highestSimilarity \textgreater= LLM\_VALIDATION\_THRESHOLD:

prompt = CreateLLMPrompt(latestTranscript, searchWindow)

AsyncCallLLM(prompt, HandleLLMResult)

return bestMatchIndex

return lastIndex
\end{minipage} \\
\midrule\noalign{}
\endhead
\bottomrule\noalign{}
\endlastfoot
\end{longtable}
}

This process ensures low latency while maintaining semantic accuracy
through adaptive matching.

\subsubsection{\texorpdfstring{\textbf{Requirement 3. Key Content
Omission
Detection}}{Requirement 3. Key Content Omission Detection}}\label{requirement-3.-key-content-omission-detection}

This function integrates directly with the flexible matching process.\\
When a skipped section is detected, the system checks whether any
omitted sentence contains a predefined \emph{key phrase} and alerts the
presenter.

\begin{enumerate}
\def\labelenumi{\arabic{enumi}.}
\item
  \textbf{Database Preparation}\\
  When users upload a script, each sentence is marked with a boolean
  attribute isKeyPhrase (true / false) and stored in the database.
\item
  \textbf{Server-Side Omission Detection}\\
  When FindCurrentPosition() identifies a new bestMatchIndex greater
  than lastIndex + 1, the server calls CheckForOmissions().
\item
  \textbf{Omission Logic}\\
  If skipped sentences are found between the two indices, each is
  inspected for isKeyPhrase == true.\\
  When detected, the omitted sentence index is flagged, and the
  following message is broadcast:\\
  \{ "action": "OMISSION\_DETECTED", "index": omittedSentenceIndex \}
\item
  \textbf{Client Notification}\\
  The frontend highlights the corresponding part of the script (e.g.,
  flashing red or adding a border) to visually warn the presenter in
  real time.\\
  Simultaneously, an asynchronous script-reconstruction task
  (Requirement 4) is triggered.
\end{enumerate}

\subsubsection{\texorpdfstring{\textbf{Requirement 4. Real-Time Script
Reconstruction}}{Requirement 4. Real-Time Script Reconstruction}}\label{requirement-4.-real-time-script-reconstruction}

This asynchronous process generates short ``bridging sentences''
whenever key content omissions are detected, ensuring smooth narrative
flow without latency in the main loop.

\begin{enumerate}
\def\labelenumi{\arabic{enumi}.}
\item
  \textbf{Asynchronous Trigger}\\
  CheckForOmissions() launches HandleOmissionAsynchronously() in a
  separate asynchronous task.
\item
  \textbf{Prompt Generation for LLM}\\
  The function combines three inputs:\\
  (a) the omitted sentence, (b) the current context sentences, and (c)
  an instruction prompt such as:\\
  \emph{``You are a presentation coach. The presenter accidentally
  omitted \textquotesingle{[}omittedSentence{]}\textquotesingle{} and is
  now moving to \textquotesingle{[}contextSentences{]}\textquotesingle.
  Please generate one short, natural bridging sentence in Korean that
  connects these topics smoothly.''}
\item
  \textbf{LLM API Invocation}\\
  The backend asynchronously requests an LLM (e.g., GPT or Gemini) to
  generate the bridging sentence.
\item
  \textbf{Response Delivery}\\
  Upon success, the server sends the message:\\
  \{ "action": "SCRIPT\_SUGGESTION", "text": llm\_generated\_sentence \}
\item
  \textbf{Client Interface Behavior}\\
  The React frontend displays the generated sentence in the AI
  Suggestion section as an alert message:\\
  ``Do you approve this suggestion?'' with \textbf{Accept (\#0064FF)}
  and \textbf{No (\#E0E6EA)} buttons.\\
  If the user selects \emph{Accept}, the updated script is applied, and
  a small alert ``💡 Update complete'' appears at the top-right corner.
\end{enumerate}
